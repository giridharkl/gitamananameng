\slcol{धृतराष्ट्र उवाच ।\\
\Index{धर्मक्षेत्रे कुरुक्षेत्रे} समवेता  युयुत्सवः।\\
मामकाः पाण्डवाश्चैव किमकुर्वत सञ्ज्य  ॥१.१॥}
\cquote{
Dhrtarastra said:
O Sanjaya, on this holy land of Kurushketra, when my people and the Pandavas have gathered eager to fight the battle, what did they do?\\}
\slcol{सञ्जय उवाच ।\\
\Index{दृष्टवा  तु   पाण्डवानीकं}   व्यूढं  दुर्योधनस्तदा ।\\
आचार्यमुपसङ्गम्य  राजा वचनमब्रवीत्  ॥१.२॥}
\cquote{Sanjaya said: After observing the Pandava army arranged in military formation, King Duryodhana approached his teacher, Dronacharya, and spoke these words.\\}
\slcol{\Index{पश्यैतां पाण्डुपुत्राणामाचार्य} महतीं चमूम् ।\\
 व्यूढां द्रुपदपुत्रेण तव शिष्येण धीमता ॥१.३॥}
\cquote{Behold, O Teacher, at this mighty army of the sons of Pandu, strategically arranged by your wise disciple, the son of Drupada.\\}

\newpage
\begin{mananam}{\mananamfont Mananam sloka - 1}
\mananamtext In my everyday life, when my body identified tendencies such as desires, anger, fear, jealousy, etc, were challenged by my deeper urge for freedom and the wisdom gathered from scriptures and teachers, which force did I choose to follow?
\end{mananam}
\WritingHand\enspace{\selfReflect\small{Self Reflection}}
\begin{inspiration}{\mananamfont Inspiration}
\mananamtext Be true to yourself and you will change for the better. Impartial observation is an essential life-skill.\\
It is not enough to simply wish for self-improvement. One must Introspect daily on their thoughts, words, and actions. 
\end{inspiration}
\newpage

\slcol{\Index{अत्र शूरा महेष्वासा} भीमार्जुनसमा युधि ।\\
युयुधानो विराटश्च द्रुपदश्च महारथः  ॥१.४॥}
\cquote{Here, in this army are great warriors like Yuyudhana, Virata, and Drupada, heroic bowmen who are equal in prowess to Bhima and Arjuna.}
\slcol{\Index{धृष्टकेतुश्चेकितानः} काशिराजश्च वीर्यवान् ।\\
पुरुजित् कुन्तिभोजश्च् शैब्यश्च नरपुङ्गवः  ॥१.५॥}
\cquote{Here are the mighty warriors - Dhrishtaketu, Chekitana, and the valiant king of Kasi; Purujit, Kuntibhoja, and the noble king Shibya.}
\slcol{\Index{युधामन्युश्च विक्रान्त} उत्तमौजाश्च वीर्यवान ।\\
सौभद्रो द्रौपदेयाश्च सर्व एव महारथाः ॥१.६॥}
\cquote{There are the brave Yudhamanyu, the valiant Uttamaujas, the son of Subhadra, and the sons of Draupadi - all great chariot-warriors.}
\slcol{\Index{अस्माकं तु विशिष्टा} ये तान्निबोध द्विजोत्तं ।\\
नायका मम सैन्यस्य संज्ञार्थं ब्रवीमि ते ॥१.७॥}
\cquote{O best among the twice-born (Dronacharya),  understand who the distinguished leaders are on our side. I name them to you for your information.}
\slcol{\Index{भवान्भीष्मश्च कर्णश्च} कृपश्च समितिञ्जयः ।\\
 अश्वत्थामा विकर्णश्च सौमदत्तिर्जयद्रथः ॥१.८॥}
\cquote{The victorious in war are Yourself, Bhishma, Karna, and Kripa. Also Ashvatthama, vikarna and Bhurisravas (the son of Somadatta).}
\slcol{\Index{अन्ये च बहवः} शूरा मदर्थे त्यक्तजीविताः ।\\
 नानाशस्त्रप्रहरणाः सर्वे युद्धविशारदाः ॥१.९॥} 
\cquote{Furthermore, there are many heroic warriors who are prepared to give up their lives for my sake. They are equipped with various weapons and are all skilled in the art of warfare.}

\newpage 
\slcol{\Index{अपर्याप्तं तदस्माकं} बलं भीष्माभिरक्षितम् ।\\
 पर्याप्तं त्विदमेतेषां बलं भीमाभिरक्षितम् ॥१.१०॥} 
\cquote{Our army, protected by Bhishma, is immeasurable. However, this army (the Pandavas’ army), protected by Bhima, is limited.}
\slcol{\Index{अयनेषु च सर्वेषु} यथाभागमवस्थिताः ।\\
 भीष्ममेवाभिरक्षन्तु भवन्तः सर्व एव हि ॥१.११॥} 
\cquote{Therefore, all of you, stationed at your respective positions, must support and protect Bhishma from all sides.}
\slcol{\Index{तस्य सञ्जनयन्हर्षं} कुरुवृद्धः पितामहः ।\\
 सिंहनादं विनद्योच्चैः शङ्खं दध्मौ प्रतापवान् ॥१.१२॥}
\cquote{Then, causing joy in Duryodhana, the mighty grandfather and eldest of the Kuru dynasty, Bhisma, roared like a lion and blew his conch shell loudly.}
\slcol{\Index{ततः शङ्खाश्च} भेर्यश्च पणवानकगोमुखाः ।\\
 सहसैवाभ्यहन्यन्त स शब्दस्तुमुलोऽभवत् ॥१.१३॥}
\cquote{Thereafter, conch shells, kettledrums, tabors (small drum), and cow-horns (a musical instrument), immediately resounded together, creating an enormous noise.}
\slcol{\Index{ततः श्वेतैर्हयैर्युक्ते} महति स्यन्दने स्थितौ ।\\
 माधवः पाण्डवश्चैव दिव्यौ शङ्खौ प्रदध्मतुः ॥१.१४॥}
\cquote{Then, Krishna (Madhava) and Arjuna (Pandava), positioned on their grand chariot pulled by white horses, blew their divine conch shells.}
\slcol{\Index{पाञ्चजन्यं हृषीकेशो} देवदत्तं धनञ्जयः ।\\
पौण्ड्रं दध्मौ महाशङ्खं भीमकर्मा वृकोदरः ॥१.१५॥}
\cquote{Hrisikesha (Krishna) blew the conch named Pancajanya, while Dhanajnaya (Arjuna) blew his conch named Devadatta. Vrikodara (Bhima), known for his great deeds, blew his mighty conch named Paundra.}
\newpage

\slcol{\Index{अनन्तविजयं राजा} कुन्तीपुत्रो युधिष्ठिरः ।\\
नकुलः सहदेवश्च सुघोषमणिपुष्पकौ ॥१.१६॥}
\cquote{King Yudhishthira, the son of Kunti, blew the conch named Anatavijaya, while Nakula and Sahadeva blew their conches named Sughosha and Manipuspaka, respectively.}
\slcol{\Index{काश्यश्च परमेष्वासः} शिखण्डी च महारथः ।\\
धृष्टद्युम्नो विराटश्च सात्यकिश्चापराजितः ॥१.१७॥}
\cquote{Kashya, the supreme archer; Sikhandi, the great warrior; Dhrishtadyumna, Virata and Satyaki (Yuyudhana), who is undefeated.} 
\slcol{\Index{द्रुपदो द्रौपदेयाश्च} सर्वशः पृथिवीपते ।\\
सौभद्रश्च महाबाहुः शङ्खान्दध्मुः पृथक्पृथक् ॥१.१८॥}
\cquote{O King of the Earth, Drupada, and the sons of Draupadi, and the son of Subhadra, one with great arms, blew their conches.}
\slcol{\Index{स घोषो धार्तराष्ट्राणां} हृदयानि व्यदारयत् ।\\
नभश्च पृथिवीं चैव तुमुलो व्यनुनादयन् ॥१.१९॥}
\cquote{The sound of these conch shells vibrated in the hearts of Dhritarashtra’s sons, and their uproarious noise echoed through the sky and the earth.}
\slcol{\Index{अथ व्यवस्थितान्दृष्ट्वा} धार्तराष्ट्रान्कपिध्वजः ।\\
प्रवृत्ते शस्त्रसम्पाते धनुरुद्यम्य पाण्डवः ॥१.२०॥\\
\Index{हृषीकेशं तदा} वाक्यमिदमाह महीपते ।}
\cquote{Then, O King, after observing Dhritarashtra’s sons positioned in military formation, Arjuna, whose flag bore the emblem of Hanuman, lifted his bow as the conflict was about to begin and spoke the following words to Hrishikesha (Krishna).}
\newpage

\slcol{अर्जुन उवाच —\\
 सेनयोरुभयोर्मध्ये रथं स्थापय मेऽच्युत ॥ १.२१ ॥\\
\Index{यावदेतान्निरीक्षेऽहं} योद्धुकामानवस्थितान् ।\\
 कैर्मया सह योद्धव्यमस्मिन्रणसमुद्यमे ॥ १.२२ ॥}
\cquote{Arjuna said: O Achyuta (Krishna), please place my chariot in between the two armies, so that I may observe the warriors positioned here for the battle, with whom I must engage with.}
\slcol{\Index{योत्स्यमानानवेक्षेऽहं} य एतेऽत्र समागताः ।\\
धार्तराष्ट्रस्य दुर्बुद्धेर्युद्धे प्रियचिकीर्षवः ॥१.२३॥}
\cquote{For, I wish to see those who have come here to fight and have assembled to support the evil-minded Duryodhana, the son of Dhritarashtra, in this battle.}
\slcol{सञ्जय उवाच —\\
\Index{एवमुक्तो हृषीकेशो} गुडाकेशेन भारत ।\\
सेनयोरुभयोर्मध्ये स्थापयित्वा रथोत्तमम् ॥१.२४॥}
\cquote{Sanjaya said: O Bharata (Dhritarashtra), after being addressed by Gudakesha (Arjuna), Hrishikesha (Krishna) placed the excellent chariot between the two armies.}
\slcol{\Index{भीष्मद्रोणप्रमुखतः} सर्वेषां च महीक्षिताम् ।\\
उवाच पार्थ पश्यैतान्समवेतान्कुरूनिति ॥१.२५॥}
\cquote{In front of Bhishma, Drona and all the other warriors of the world, Krishna said “O Partha (Arjuna), look at these Kurus who have assembled here”.}
\slcol{\Index{तत्रापश्यत्स्थितान्पार्थः} पितॄनथ पितामहान् ।\\
आचार्यान्मातुलान्भ्रातॄन्पुत्रान्पौत्रान्सखींस्तथा ॥१.२६॥}
\cquote{Then, Partha (Arjuna) saw on the battlefield, his own relatives, including his fathers, grandfathers, teachers, maternal uncles, brothers, sons, grandsons, and friends, all standing there.}
\newpage

\slcol{\Index{श्वशुरान्सुहृदश्चैव}सेनयोरुभयोरपि ।\\
 तान्समीक्ष्य स कौन्तेयः सर्वान्बन्धूनवस्थितान् ॥१.२७॥\\
\Index{कृपया परयाविष्टो} विषीदन्निदमब्रवीत् ।}
\cquote{Seeing his fathers-in-law, friends, and relatives positioned on both sides of the battlefield, Arjuna, overwhelmed with deep compassion and sorrow, spoke the following words.}
\slcol{अर्जुन उवाच —\\ 
दृष्ट्वेमान्स्वजनान्कृष्ण युयुत्सून्समुपस्थितान् ॥१.२८॥}
\cquote{Arjuna said: O Krishna, after seeing these relatives and friends who desire to fight, my mind is overwhelmed with sorrow and compassion.}
\slcol{\Index{सीदन्ति मम गात्राणि} मुखं च परिशुष्यति ।\\
वेपथुश्च शरीरे मे रोमहर्षश्च जायते ॥१.२९॥}
\cquote{My limbs are failing, my face is drying up, my body trembles, and my hair stands on end.}
\slcol{\Index{गाण्डीवं स्रंसते} हस्तात्त्वक्चैव परिदह्यते ।\\
न च शक्नोम्यवस्थातुं भ्रमतीव च मे मनः ॥१.३०॥}
\cquote{O Krishna, seeing my people, gathered here for battle, my limbs fail, and my mouth is parched. My body shivers, and my hair stands on end. My Gandiva-bow slips from my grip, and my skin burns as my mind spins in confusion.}
\slcol{\Index{निमित्तानि च पश्यामि} विपरीतानि केशव ।\\
 न च श्रेयोऽनुपश्यामि हत्वा स्वजनमाहवे ॥१.३१॥}
\cquote{O Keshava (Krishna), I see only inauspicious omens and adverse signs. I do not perceive any benefit in killing my own kin in this battle.}
\slcol{\Index{न काङ्क्षे विजयं} कृष्ण न च राज्यं सुखानि च ।\\
किं नो राज्येन गोविन्द किं भोगैर्जीवितेन वा ॥१.३२॥}
\cquote{O Krishna, I desire neither victory, nor kingdom, nor happiness or pleasures. O Govinda, of what use is the kingdom to us? What use are pleasures or even life itself?}
\newpage
\begin{mananam}{\mananamfont Mananam sloka - 30}
\mananamtext Let me reflect on those moments in life when I had to face tremendous anxiety and was overwhelmed by my outward circumstances. In such situations, was I aware that my physical condition was paralysed due to my mental fears?  
\end{mananam}
\WritingHand\enspace{\selfReflect\small{Self Reflection}}
\begin{inspiration}{\mananamfont Inspiration}
\mananamtext Remain vigilant of your thoughts.Your mental states impact your body.
\end{inspiration}
\newpage

\slcol{\Index{येषामर्थे काङ्क्षितं} नो राज्यं भोगाः सुखानि च ।\\
त इमेऽवस्थिता युद्धे प्राणांस्त्यक्त्वा धनानि च ॥१.३३॥}
\cquote{For whose sake we desire the kingdom, pleasures, and happiness, they are all standing here, prepared for battle, having renounced their lives and wealth.}
\slcol{\Index{आचार्याः पितरः} पुत्रास्तथैव च पितामहाः ।\\
मातुलाः श्वशुराः पौत्राः स्यालाः सम्बन्धिनस्तथा ॥१.३४॥}
\cquote{Teachers, fathers, sons, grandfathers, maternal uncles, fathers-in-law, grandsons, brothers-in-law, and other relatives.}
\slcol{\Index{एतान्न हन्तुमिच्छामि} घ्नतोऽपि मधुसूदन ।\\
अपि त्रैलोक्यराज्यस्य हेतोः किं नु महीकृते ॥१.३५॥}
\cquote{I do not desire to kill them, even if they attack me, O Madhusudana (Krishna). I have no desire to fight for the sake of the kingdom, even if all the happiness of the three worlds were offered to me. What joy would there be in killing the sons of Dhritarashtra?}
\slcol{\Index{निहत्य धार्तराष्ट्रान्नः} का प्रीतिः स्याज्जनार्दन ।\\
पापमेवाश्रयेदस्मान्हत्वैतानाततायिनः ॥१.३६॥}
\cquote{O Janardana (Krishna), what pleasure will we derive from killing the sons of Dhritarashtra? By killing these wrongdoers, sin alone will befall to us.}
\slcol{\Index{तस्मान्नार्हा वयं हन्तुं} धार्तराष्ट्रान्सबान्धवान् । \\
स्वजनं हि कथं हत्वा सुखिनः स्याम माधव ॥१.३७॥}
\cquote{Therefore, we cannot destroy our own relatives, the sons of Dhritarashtra, our relatives. How can we be happy, O Madhava, by killing our own kin?}
\slcol{\Index{यद्यप्येते न पश्यन्ति} लोभोपहतचेतसः ।\\
कुलक्षयकृतं दोषं मित्रद्रोहे च पातकम् ॥१.३८॥}
\cquote{Even if these people, whose minds are veiled by greed, fail to recognize the wrongdoing in the destruction of the family and the sin of betraying friends.}

\newpage
\begin{mananam}{\mananamfont Mananam sloka - 37}
\mananamtext Have there been times when I justified my actions or inactions due to lack of responsibility? Am I afraid to disassociate myself from those who pull me downward spiritually and negatively influence me, out of a false sense of sympathy?
\end{mananam}
\WritingHand\enspace{\selfReflect\small{Self Reflection}}
\begin{inspiration}{\mananamfont Inspiration}
\mananamtext Rise up to the challenge of life. Take responsibility for your own actions and choices. Mercilessly, get rid of all your negative habits and influences.
\end{inspiration}
\newpage

\slcol{\Index{कथं न ज्ञेयमस्माभिः} पापादस्मान्निवर्तितुम् ।\\
कुलक्षयकृतं दोषं प्रपश्यद्भिर्जनार्दन ॥१.३९॥}
\cquote{O Janardana (Krishna), how can we fail to recognize that we must avoid this sin? We are witnessing the fault of destroying the family.}
\slcol{\Index{कुलक्षये प्रणश्यन्ति} कुलधर्माः सनातनाः ।\\
धर्मे नष्टे कुलं कृत्स्नमधर्मोऽभिभवत्युत ॥१.४०॥}
\cquote{When a family declines, its traditional practices are lost. When righteousness is lost, the entire family is overwhelmed by unrighteousness.}
\slcol{\Index{अधर्माभिभवात्कृष्ण} प्रदुष्यन्ति कुलस्त्रियः ।\\
स्त्रीषु दुष्टासु वार्ष्णेय जायते वर्णसङ्करः ॥१.४१॥}
\cquote{O Krishna, when unrighteousness prevails, the women of the family become corrupted. O Varshneya (descendant of the Vrishni clan), this causes chaos in the caste system (varnasankara) and disrupts the social order.}
\slcol{\Index{सङ्करो नरकायैव} कुलघ्नानां कुलस्य च ।\\
पतन्ति पितरो ह्येषां लुप्तपिण्डोदकक्रियाः ॥१.४२॥}
\cquote{Such a mix-up leads to hell for those who ruin the family traditions. The ancestors of those who have neglected the rites of offering food and water suffer and fall from grace.}
\slcol{\Index{दोषैरेतैः कुलघ्नानां} वर्णसङ्करकारकैः ।\\
उत्साद्यन्ते जातिधर्माः कुलधर्माश्च शाश्वताः ॥१.४३॥}
\cquote{These evil actions, caused by those who destroy families and create disorder in the castes, result in the destruction of traditional duties and timeless family values.}
\slcol{\Index{उत्सन्नकुलधर्माणां} मनुष्याणां जनार्दन ।\\
 नरके नियतं वासो भवतीत्यनुशुश्रुम ॥१.४४॥}
\cquote{O Janardana (Krishna), we have heard that those who destroy families are destined to dwell in hell permanently. This fate is inevitable for those who bring down their family values.} 
\slcol{\Index{अहो बत महत्पापं} कर्तुं व्यवसिता वयम् ।\\
 यद्राज्यसुखलोभेन हन्तुं स्वजनमुद्यताः ॥१.४५॥}
\cquote{Alas! We are about to commit a great sin. We are determined to kill our own kinsmen in order to enjoy the pleasures ruling the kingdom.}
\slcol{\Index{यदि मामप्रतीकारमशस्त्रं} शस्त्रपाणयः ।\\
धार्तराष्ट्रा रणे हन्युस्तन्मे क्षेमतरं भवेत् ॥१.४६॥}
\cquote{If the armed sons of Dhritarashtra were to kill me unarmed and defenseless in the battle, it would be more favorable for me.}
\slcol{सञ्जय उवाच —\\
\Index{एवमुक्त्वार्जुनः सं‍ख्ये} रथोपस्थ उपाविशत् ।\\
विसृज्य सशरं चापं शोकसंविग्नमानसः ॥१.४७॥}
\cquote{Sanjaya said: Thus, speaking these words, Arjuna, overwhelmed with sorrow and dejection on the battlefield, took a seat in his chariot and set aside his bow and arrow.}

\begin{center}
इति श्रीमद्भगवद्गीतासूपनिषत्सु  ब्रह्मविद्यायां योगशास्त्रे\\ श्रीकृष्णर्जुनसंवादे अर्जुनविषादयोगो  नाम प्रथमोऽध्यायः ॥
\end{center}




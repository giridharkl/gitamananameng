\begin{center}
\large\devfont दिशन्तु शं मे गुरुपादपांसवः
\end{center}

{\small
Srimad Bhagavad Gita is a unique and unparalleled jewel among all scriptures. It serves as a guiding light not only for renunciants but also for those laden with worldly responsibilities, attempting to balance the material with the spiritual. It advocates them to be impartial in the dealings of their body and mind, which out of ignorance they get identified with. It is a wrong notion that in order to accomplish one's duties in life, a sense of attachment is necessary. Lord Krishna, the biggest संसारी (samsari) of all is a perfect unattached असंसारी (asamsari). He demonstrates through his actions and words as to how to function in this tangible, material world while being grounded in the sublime bliss.\\


To attain this state, one needs proper guidance from capable teachers who show the path that initially appears to be conflicting and full of challenges. As pointed out by the deep insight of the author of this master guide book  in verses 52-53 of Ch 2:
"The purpose of the teacher and scriptures is to shake us out of delusion. Confusions and challenges are a part of spiritual journey as we let go of our worldly thinking patterns and seek refuge in the higher reality. As we thus progress, we begin to gain clarity on the path."\\


This book leads the readers, through the author's revolutionary contemplations, from body to mind and from mind to consciousness, penetrating the layers of one's existence. The introspective questions under the title Mananam in each section give no respite to those who are engaged in self appeasement and self-deceit (आत्म प्रवञ्चना) and challenges them towards a honest self-assessment. On the other hand, the words under Inspiration serves as an inexhaustable source of positivity making the arduous and confusing spiritual path relatively easier.\\


This handbook gives ample material for aspirants to find solution to the various issues of life, not outwardly but within oneself. The content of this book shows the profound understanding of the author on the human psychology, progressing first to a positive mental state and then beyond it, to the innermost eternal Self. As the readers of this book embark on this journey, they are certain to surpass the domain of human mind and arrive at the अधिष्ठान चैतन्यम् (ultimate substratum of sentiency), by the grace of the singer of this celestial song, "भगवद्गीता".\\


May the selfless effort of Swami Nirgunanda Giri ji in penning his thougts for the benefit of the seekers be blessed by the Eternal Teacher, Lord Krishna.\\
}
\begin{center}
\devfont मङ्गलम्  सर्वम्
\end{center}

{\large\devfont स्वामी स्वानन्द तीर्थ\\}
\devfont आचार्य, कैलास आश्रम\\
\devfont ऋषिकेश - उत्तराखण्ड
\newpage

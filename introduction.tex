{\small Each of us must face our battles of life. On the Battlefield of Kurukshetra, Lord Krishna gave to his anguished disciple Arjuna, the most sacred teachings of finding the spiritual in the mundane, in succinct and practical ways. These sublime teachings, the essence of Upanishads, has been handed down to us by Sage Vyasa in the format of ‘Bhagavad Gita, the Divine song.’\\


Our circumstances may be different from that of Arjuna and so also our challenges. But the universal teachings of Gita present every seeker of truth with a model for self-improvement and spiritual progress. 
The teachings of Gita are not just a spiritual teaching meant for dedicated spiritual seekers but a handbook for life. Anyone seeking to balance work and family responsibilities with mental peace and freedom from stress will find these teachings valuable.\\


This Diary and Journal book based is not an attempt to impart the teachings of Gita nor to provide any commentaries as several Gurus and scholars have already done that. This Gita Mananam Journal is an attempt to allow you to contemplate on the teachings and make it your own. The Sanskrit word Mananam, is used in this context as a process of contemplation on what has already been heard (Shravanam) or read.\\


So this Journal is designed to give you the space to express yourself and apply the teachings in your life. The questions on the Gita shlokas are framed to allow you to find personal meaning within the context of these teachings and thereby find greater clarity within your life circumstance. \\


We hope that as Lord Krishna inspired Arjuna to fight his righteous battle, so may the Lord inspire you through this work, in your own daily duties of life. To accomplish your duties successfully without neglecting your inner peace, personal progress and to remain anchored in the Divine, is the eternal purpose of this Celestial Song. \\}

स्वामि निर्गुणानन्द गिरि